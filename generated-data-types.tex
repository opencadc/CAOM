
% -------------------------------------------
% Items to substitute into the ivoatex document template.
%
%\ivoagroup{Data Model Working Group}

%\title{DataTypes}


%\author{Patrick Dowler}
    
%\author{Canadian Astronomy Data Centre}
    
% -------------------------------------------

\pagebreak
\section{Model: dt }
  
  % INSERT FIGURE HERE
  %\begin{figure}[h]
  %\begin{center}
  %  \includegraphics[width=\textwidth]{????.png}
  %  \caption{???}\label{fig:????}
  %\end{center}
  %\end{figure}

  This model is a general purpose collection of data types required for CAOM but reusable in other data models.

  \subsection{Circle}
  \label{sect:Circle}
    a circular region in two-dimensions

    \subsubsection{Circle.center}
      \textbf{vodml-id: Circle.center} \newline
      \textbf{type: \hyperref[sect:Point]{dt:Point}} \newline
      \textbf{multiplicity: 1} \newline
      [TODO add description!]

    \subsubsection{Circle.radius}
      \textbf{vodml-id: Circle.radius} \newline
      \textbf{type: \hyperref[sect:double]{dt:double}} \newline
      \textbf{multiplicity: 1} \newline
      [TODO add description!]

  \subsection{Interval}
  \label{sect:Interval}
    a set of numeric values defined by a lower and upper bound (bounds included: [a,b])

    \subsubsection{Interval.lower}
      \textbf{vodml-id: Interval.lower} \newline
      \textbf{type: \hyperref[sect:double]{dt:double}} \newline
      \textbf{multiplicity: 1} \newline
      [TODO add description!]

    \subsubsection{Interval.upper}
      \textbf{vodml-id: Interval.upper} \newline
      \textbf{type: \hyperref[sect:double]{dt:double}} \newline
      \textbf{multiplicity: 1} \newline
      [TODO add description!]

  \subsection{MultiShape}
  \label{sect:MultiShape}
    multiple simple shapes describing regions in two-dimensions

    \subsubsection{MultiShape.shapes}
      \textbf{vodml-id: MultiShape.shapes} \newline
      \textbf{type: \hyperref[sect:Shape]{dt:Shape}} \newline
      \textbf{multiplicity: 1..*} \newline
      [TODO add description!]

  \subsection{Point}
  \label{sect:Point}
    location in two-dimensions

    \subsubsection{Point.cval1}
      \textbf{vodml-id: Point.cval1} \newline
      \textbf{type: \hyperref[sect:double]{dt:double}} \newline
      \textbf{multiplicity: 1} \newline
      [TODO add description!]

    \subsubsection{Point.cval1}
      \textbf{vodml-id: Point.cval2} \newline
      \textbf{type: \hyperref[sect:double]{dt:double}} \newline
      \textbf{multiplicity: 1} \newline
      [TODO add description!]

  \subsection{Polygon}
  \label{sect:Polygon}
    a simple polygon region in two-dimensions defined a sequence of points

    \subsubsection{Polygon.points}
      \textbf{vodml-id: Polygon.points} \newline
      \textbf{type: \hyperref[sect:Point]{dt:Point}} \newline
      \textbf{multiplicity: 3..*} \newline
      [TODO add description!]

  \subsection{Shape (Abstract)}
  \label{sect:Shape}
    [TODO add description!]

  \subsection{uuid}
  \label{sect:uuid}
  a 128-bit globally unique binary identifier

  \subsection{double}
  \label{sect:double}
  an IEEE double precision (64-bit) floating point value

  \subsection{int32}
  \label{sect:int32}
  a 32-bit signed integer

  \subsection{int64}
  \label{sect:int64}
  a 64-bit signed integer