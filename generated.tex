
% -------------------------------------------
% Items to substitute into the ivoatex document template.
%
%\ivoagroup{Data Model Working Group}

%\title{Common Archive Observation Model}


%\author{Patrick Dowler}
    
%\author{Canadian Astronomy Data Centre}
    
% -------------------------------------------

\pagebreak
\section{Model: caom2 }
  
  % INSERT FIGURE HERE
  %\begin{figure}[h]
  %\begin{center}
  %  \includegraphics[width=\textwidth]{????.png}
  %  \caption{???}\label{fig:????}
  %\end{center}
  %\end{figure}

  a general purpose data model for use as the core data model of an astronomical data centre

  \subsection{Algorithm}
  \label{sect:Algorithm}
    the algorithm that was responsible for creating the observation; for a DerivedObservation this is the algorithm that defines the intended set of members to include

    \subsubsection{Algorithm.name}
      \textbf{vodml-id: Algorithm.name} \newline
      \textbf{type: \hyperref[sect:ivoa]{ivoa:string}} \newline
      \textbf{multiplicity: 1} \newline
      common name of the algorithm; the value 'exposure' is reserved for use in SimpleObservation; TBD: publish a list of acceptable values as a machine-readable vocabulary?

  \subsection{Artifact}
  \label{sect:Artifact}
    a physical product (typically a file)

    \subsubsection{Artifact.uri}
      \textbf{vodml-id: Artifact.uri} \newline
      \textbf{type: \hyperref[sect:ivoa]{ivoa:anyURI}} \newline
      \textbf{multiplicity: 1} \newline
      an identifier that resolves to the storage location of the artifact

    \subsubsection{Artifact.productType}
      \textbf{vodml-id: Artifact.productType} \newline
      \textbf{type: \hyperref[sect:DataLinkSemantics]{caom2:DataLinkSemantics}} \newline
      \textbf{multiplicity: 1} \newline
      the relationship of this Artifact to the parent Plane (DataLink semantics)

    \subsubsection{Artifact.releaseType}
      \textbf{vodml-id: Artifact.releaseType} \newline
      \textbf{type: \hyperref[sect:ReleaseType]{caom2:ReleaseType}} \newline
      \textbf{multiplicity: 1} \newline
      field indicating how access permissions for this artifact are determined

    \subsubsection{Artifact.contentType}
      \textbf{vodml-id: Artifact.contentType} \newline
      \textbf{type: \hyperref[sect:ivoa]{ivoa:string}} \newline
      \textbf{multiplicity: 0..1} \newline
      label specifying the format of the resolved artifact; typically a MIME-type

    \subsubsection{Artifact.contentLength}
      \textbf{vodml-id: Artifact.contentLength} \newline
      \textbf{type: \hyperref[sect:ivoa]{ivoa:integer}} \newline
      \textbf{multiplicity: 0..1} \newline
      the size of the resolved artifact; typically file size in bytes

    \subsubsection{Artifact.contentChecksum}
      \textbf{vodml-id: Artifact.contentChecksum} \newline
      \textbf{type: \hyperref[sect:ivoa]{ivoa:anyURI}} \newline
      \textbf{multiplicity: 0..1} \newline
      the checksum of the artifact data; the URI must conform to the pattern {algorithm}:{value}, for example: md5:4be91751541fd804e7207663a0822f56 (NEW in CAOM-2.3)

    \subsubsection{Artifact.contentRelease}
      \textbf{vodml-id: Artifact.contentRelease} \newline
      \textbf{type: \hyperref[sect:ivoa]{ivoa:datetime}} \newline
      \textbf{multiplicity: 0..1} \newline
      timestamp after which content for the plane is public (new in CAOM-2.4); if set, this value overrides the permission implied by the releaseType and Plane release dates

    \subsubsection{Artifact.contentReadGroups}
      \textbf{vodml-id: Artifact.contentReadGroups} \newline
      \textbf{type: \hyperref[sect:ivoa]{ivoa:anyURI}} \newline
      \textbf{multiplicity: 0..*} \newline
      list of groups (of users) that are allowed to access the content of this artifact; this is applicable when the effective release date is null or in the future (new in CAOM-2.4)

    \subsubsection{Artifact.descriptionID}
      \textbf{vodml-id: Artifact.descriptionID} \newline
      \textbf{type: \hyperref[sect:ivoa]{ivoa:anyURI}} \newline
      \textbf{multiplicity: 0..1} \newline
      identifier for an ArtifactDescription entity (new in CAOM-2.5)

    \subsubsection{Artifact.parts}
      \textbf{vodml-id: Artifact.parts} \newline
      \textbf{type: \hyperref[sect:Part]{caom2:Part}} \newline
      \textbf{multiplicity: 0..*} \newline
      the component parts of this artifact

  \subsection{ArtifactDescription}
  \label{sect:ArtifactDescription}
    shared and referenceable description of an artitfact (NEW in CAOM-2.5)

    \subsubsection{ArtifactDescription.uri}
      \textbf{vodml-id: ArtifactDescription.uri} \newline
      \textbf{type: \hyperref[sect:ivoa]{ivoa:anyURI}} \newline
      \textbf{multiplicity: 1} \newline
      logical identifier for this description

    \subsubsection{ArtifactDescription.description}
      \textbf{vodml-id: Artifact.description} \newline
      \textbf{type: \hyperref[sect:ivoa]{ivoa:string}} \newline
      \textbf{multiplicity: 1} \newline
      description of an artifact

  \subsection{CaomEntity (Abstract)}
  \label{sect:CaomEntity}
    base CAOM entity class to support accumulated metadata checkcums and lastModified timestamps to support synchronisation

    \subsubsection{CaomEntity.maxLastModified}
      \textbf{vodml-id: CaomEntity.maxLastModified} \newline
      \textbf{type: \hyperref[sect:ivoa]{ivoa:datetime}} \newline
      \textbf{multiplicity: 0..1} \newline
      maximum timestamp of last modification of this entity and all child entities; the timestamp is intended to be applied or updated when the entity is stored (e.g. in a database)

    \subsubsection{CaomEntity.accMetaChecksum}
      \textbf{vodml-id: CaomEntity.accMetaChecksum} \newline
      \textbf{type: \hyperref[sect:ivoa]{ivoa:anyURI}} \newline
      \textbf{multiplicity: 0..1} \newline
      accumulated checksum of the metadata of this entity and all child entities; (NEW in CAOM-2.3) The URI must conform to the pattern {algorithm}:{value}, for example: md5:4be91751541fd804e7207663a0822f56. The accumulated checksum of an entity is computed by accumulating the byte representation of entity checksums in the following order: (1) the metaChecksum of the current entity, (2) the accMetaChecksum of all child entities accumulated in order of the child's Entity.id. For an entity with no children, the accMetaChecksum is derived only from the metaChecksum but it is not equal to it because it is a checksum of that checksum and not a checksum of the same metadata directly.

  \subsection{Chunk}
  \label{sect:Chunk}
    a quantitatively defined subsection of a data array; the part contains the whole array

    \subsubsection{Chunk.productType}
      \textbf{vodml-id: Chunk.productType} \newline
      \textbf{type: \hyperref[sect:DataLinkSemantics]{caom2:DataLinkSemantics}} \newline
      \textbf{multiplicity: 0..1} \newline
      the relationship of this Artifact to the parent Plane (DataLink semantics)

    \subsubsection{Chunk.naxis}
      \textbf{vodml-id: Chunk.naxis} \newline
      \textbf{type: \hyperref[sect:ivoa]{ivoa:integer}} \newline
      \textbf{multiplicity: 0..1} \newline
      number of axes in the data array; value must be in [1,7] since CAOM supports a maximum of seven axes; furthermore, if naxis has a value, the axis index values 1 to {naxis} must be assigned (to positionAxis1, positionAxis2, energyAxis, timeAxis, polarizationAxis, customAxis, and/or observableAxis) and each axis index value assigned to those fields must be unique; axis index values above {naxis} may be assigned in order to preserve the ordering of metadata-only WCS (see FITS-WCS WCSAXES defintion for an example of how this could originate and be used)

    \subsubsection{Chunk.positionAxis1}
      \textbf{vodml-id: Chunk.positionAxis1} \newline
      \textbf{type: \hyperref[sect:ivoa]{ivoa:integer}} \newline
      \textbf{multiplicity: 0..1} \newline
      index of the first position axis; if set: positionAxis2 and position must also have values

    \subsubsection{Chunk.positionAxis2}
      \textbf{vodml-id: Chunk.positionAxis2} \newline
      \textbf{type: \hyperref[sect:ivoa]{ivoa:integer}} \newline
      \textbf{multiplicity: 0..1} \newline
      index of the second position axis; if set: positionAxis1 and position must also have values

    \subsubsection{Chunk.energyAxis}
      \textbf{vodml-id: Chunk.energyAxis} \newline
      \textbf{type: \hyperref[sect:ivoa]{ivoa:integer}} \newline
      \textbf{multiplicity: 0..1} \newline
      index of the energy axis; if set: energy must have a value

    \subsubsection{Chunk.timeAxis}
      \textbf{vodml-id: Chunk.timeAxis} \newline
      \textbf{type: \hyperref[sect:ivoa]{ivoa:integer}} \newline
      \textbf{multiplicity: 0..1} \newline
      index of the time axis; if set: time must have a value

    \subsubsection{Chunk.polarizationAxis}
      \textbf{vodml-id: Chunk.polarizationAxis} \newline
      \textbf{type: \hyperref[sect:ivoa]{ivoa:integer}} \newline
      \textbf{multiplicity: 0..1} \newline
      index of the polarization axis; if set: polarization must have a value

    \subsubsection{Chunk.customAxis}
      \textbf{vodml-id: Chunk.customAxis} \newline
      \textbf{type: \hyperref[sect:ivoa]{ivoa:integer}} \newline
      \textbf{multiplicity: 0..1} \newline
      index of the custom axis; if set: custom must have a value

    \subsubsection{Chunk.observableAxis}
      \textbf{vodml-id: Chunk.observableAxis} \newline
      \textbf{type: \hyperref[sect:ivoa]{ivoa:integer}} \newline
      \textbf{multiplicity: 0..1} \newline
      index of the observable axis; if set: observable must have a value

    \subsubsection{Chunk.position}
      \textbf{vodml-id: Chunk.position} \newline
      \textbf{type: \hyperref[sect:SpatialWCS]{caom2:SpatialWCS}} \newline
      \textbf{multiplicity: 0..1} \newline
      spatial WCS description of the data array or subsection thereof; if positionAxis1 and positionAxis2 indices are null or have values above {naxis}, then the spatial WCS is metadata only (usually the degenerate case of one spatial pixel)

    \subsubsection{Chunk.energy}
      \textbf{vodml-id: Chunk.energy} \newline
      \textbf{type: \hyperref[sect:SpectralWCS]{caom2:SpectralWCS}} \newline
      \textbf{multiplicity: 0..1} \newline
      spectral WCS description of the data array or subsection thereof; if energyAxis index is null or has a value above {naxis}, then the spectral WCS is metadata only (usually the degenerate case of one spectral pixel)

    \subsubsection{Chunk.time}
      \textbf{vodml-id: Chunk.time} \newline
      \textbf{type: \hyperref[sect:TemporalWCS]{caom2:TemporalWCS}} \newline
      \textbf{multiplicity: 0..1} \newline
      temporal WCS description of the data array or subsection thereof; if timeAxis index is null or has a value above {naxis}, then the time WCS is metadata only (usually the degenerate case of one time pixel for the whole data array)

    \subsubsection{Chunk.polarization}
      \textbf{vodml-id: Chunk.polarization} \newline
      \textbf{type: \hyperref[sect:PolarizationWCS]{caom2:PolarizationWCS}} \newline
      \textbf{multiplicity: 0..1} \newline
      polarization WCS description of the data array or subsection thereof; if polarizationAxis index is null or has a value above {naxis}, then the polarization WCS is metadata only (usually the degenerate case of one polarization pixel)

    \subsubsection{Chunk.custom}
      \textbf{vodml-id: Chunk.custom} \newline
      \textbf{type: \hyperref[sect:CustomWCS]{caom2:CustomWCS}} \newline
      \textbf{multiplicity: 0..1} \newline
      custom WCS description of the data array or subsection thereof; if customAxis index is null or has a value above {naxis}, then the custom WCS is metadata only (usually the degenerate case of one custom pixel)

    \subsubsection{Chunk.observable}
      \textbf{vodml-id: Chunk.observable} \newline
      \textbf{type: \hyperref[sect:ObservableAxis]{caom2:ObservableAxis}} \newline
      \textbf{multiplicity: 0..1} \newline
      observable description of the data array or subsection thereof; if observableAxis index is null or has a value above {naxis}, then the observable axis is metadata only (usually the degenerate case of one observable for the whole data array)

  \subsection{CustomAxis}
  \label{sect:CustomAxis}
    description of a custom coordinate axis (new in CAOM-2.4)

    \subsubsection{CustomAxis.ctype}
      \textbf{vodml-id: CustomAxis.ctype} \newline
      \textbf{type: \hyperref[sect:ivoa]{ivoa:string}} \newline
      \textbf{multiplicity: 1} \newline
      coordinate type code

    \subsubsection{CustomAxis.bounds}
      \textbf{vodml-id: CustomAxis.bounds} \newline
      \textbf{type: \hyperref[sect:types.Interval]{caom2:types.Interval}} \newline
      \textbf{multiplicity: 1} \newline
      custom coordinate coverage (cardinality changed in CAOM-2.5)

    \subsubsection{CustomAxis.samples}
      \textbf{vodml-id: CustomAxis.samples} \newline
      \textbf{type: \hyperref[sect:types.Interval]{caom2:types.Interval}} \newline
      \textbf{multiplicity: 1..*} \newline
      detailed custom coordinate coverage (refactored in CAOM-2.5)

    \subsubsection{CustomAxis.dimension}
      \textbf{vodml-id: CustomAxis.dimension} \newline
      \textbf{type: \hyperref[sect:ivoa]{ivoa:integer}} \newline
      \textbf{multiplicity: 0..1} \newline
      number of discrete samples (pixels) along custom axis

  \subsection{DataLinkSemantics}
  \label{sect:DataLinkSemantics}
    term from the DataLink Core vocabulary (CHANGED from ProductType enumeration in CAOM-2.5)

  \subsection{DataProductType}
  \label{sect:CalibrationStatus}
    term from the IVOA calibration-status vocabulary; this vocabulary does not yet exist but can be created using optional terms in the ObsCore standard (NEW in CAOM-2.5)

  \subsection{DataProductType}
  \label{sect:DataProductType}
    term from the IVOA product-type vocabulary (CHANGED from CAOM DataProductType vocabulary based on ObsCore in CAOM-2.5)

  \subsection{DataQuality}
  \label{sect:DataQuality}
    description of the data quality

    \subsubsection{DataQuality.flag}
      \textbf{vodml-id: DataQuality.flag} \newline
      \textbf{type: \hyperref[sect:Quality]{caom2:Quality}} \newline
      \textbf{multiplicity: 1} \newline
      flag indicating the data quality

  \subsection{DerivedObservation}
  \label{sect:DerivedObservation}
    an observation derived from one or more observations (name and intent changed in CAOM-2.4)

    \subsubsection{DerivedObservation.members}
      \textbf{vodml-id: DerivedObservation.members} \newline
      \textbf{type: \hyperref[sect:ivoa]{ivoa:anyURI}} \newline
      \textbf{multiplicity: 0..*} \newline
      members are the observations grouped together by the algorithm that defines the derivation; these are the intended components of the composite product -- actual inputs are described by the provenance; members may be simple or derived observations (arbitrary heirarchy); a derived observation made by combining multiple observations is equivalent to a composite observation (CAOM-2.3 and older); derived observations with one or more members may be defined such that they only include a subset of each member (they are extracted from the progenitor)

  \subsection{Energy}
  \label{sect:Energy}
    description of the energy coverage and sampling of the data

    \subsubsection{Energy.bounds}
      \textbf{vodml-id: Energy.bounds} \newline
      \textbf{type: \hyperref[sect:types.Interval]{caom2:types.Interval}} \newline
      \textbf{multiplicity: 1} \newline
      simple outer bounds of energy coverage of the data (cardinality changed in CAOM-2.5)

    \subsubsection{Energy.samples}
      \textbf{vodml-id: Energy.samples} \newline
      \textbf{type: \hyperref[sect:types.Interval]{caom2:types.Interval}} \newline
      \textbf{multiplicity: 1..*} \newline
      detailed energy coverage of the data (refactored in CAOM-2.5)

    \subsubsection{Energy.energyBands}
      \textbf{vodml-id: Energy.energyBands} \newline
      \textbf{type: \hyperref[sect:EnergyBand]{caom2:EnergyBand}} \newline
      \textbf{multiplicity: 0..*} \newline
      standard name of the energy regime(s) included in the data (attribute name and cardinality changed in CAOM-2.4)

    \subsubsection{Energy.dimension}
      \textbf{vodml-id: Energy.dimension} \newline
      \textbf{type: \hyperref[sect:ivoa]{ivoa:integer}} \newline
      \textbf{multiplicity: 0..1} \newline
      number of measurements (pixels) on the energy axis

    \subsubsection{Energy.resolvingPower}
      \textbf{vodml-id: Energy.resolvingPower} \newline
      \textbf{type: \hyperref[sect:ivoa]{ivoa:real}} \newline
      \textbf{multiplicity: 0..1} \newline
      mean spectral resolving power per pixel (relative resolution)

    \subsubsection{Energy.resolvingPowerBounds}
      \textbf{vodml-id: Energy.resolvingPowerBounds} \newline
      \textbf{type: \hyperref[sect:types.Interval]{caom2:types.Interval}} \newline
      \textbf{multiplicity: 0..1} \newline
      range of resolving power within the bounds (relative resolution)

    \subsubsection{Energy.resolution}
      \textbf{vodml-id: Energy.resolution} \newline
      \textbf{type: \hyperref[sect:ivoa]{ivoa:real}} \newline
      \textbf{multiplicity: 0..1} \newline
      mean absolute spectral resolution per pixel (new in CAOM-2.5)

    \subsubsection{Energy.resolutionBounds}
      \textbf{vodml-id: Energy.resolutionBounds} \newline
      \textbf{type: \hyperref[sect:types.Interval]{caom2:types.Interval}} \newline
      \textbf{multiplicity: 0..1} \newline
      range of absolute spectral resolution within the bounds (new in CAOM-2.5)

    \subsubsection{Energy.sampleSize}
      \textbf{vodml-id: Energy.sampleSize} \newline
      \textbf{type: \hyperref[sect:ivoa]{ivoa:real}} \newline
      \textbf{multiplicity: 0..1} \newline
      mean pixel size

    \subsubsection{Energy.bandpassName}
      \textbf{vodml-id: Energy.bandpassName} \newline
      \textbf{type: \hyperref[sect:ivoa]{ivoa:string}} \newline
      \textbf{multiplicity: 0..*} \newline
      telescope- and instrument-specific name for the energy band(s) included; multiple bands may be included if energies from all specified bands are included in the data (in the sense of union)

    \subsubsection{Energy.transition}
      \textbf{vodml-id: Energy.transition} \newline
      \textbf{type: \hyperref[sect:EnergyTransition]{caom2:EnergyTransition}} \newline
      \textbf{multiplicity: 0..1} \newline
      target energy transition for this data

    \subsubsection{Energy.restwav}
      \textbf{vodml-id: Energy.rest} \newline
      \textbf{type: \hyperref[sect:ivoa]{ivoa:real}} \newline
      \textbf{multiplicity: 0..1} \newline
      rest energy of the target energy transition (name changed in CAOM-2.5)

    \subsubsection{Energy.calibration}
      \textbf{vodml-id: Energy.calibration} \newline
      \textbf{type: \hyperref[sect:CalibrationStatus]{caom2:CalibrationStatus}} \newline
      \textbf{multiplicity: 0..1} \newline
      term describing the calibration of the energy axis values (new in CAOM-2.5)

  \subsection{Entity (Abstract)}
  \label{sect:Entity}
    base entity class to support persistence; entity attributes are generally set or updated by persistence implementations

    \subsubsection{Entity.id}
      \textbf{vodml-id: Entity.id} \newline
      \textbf{type: \hyperref[sect:uuid]{caom2:uuid}} \newline
      \textbf{multiplicity: 1} \newline
      globally unique identifier (primary key)

    \subsubsection{Entity.lastModified}
      \textbf{vodml-id: Entity.lastModified} \newline
      \textbf{type: \hyperref[sect:ivoa]{ivoa:datetime}} \newline
      \textbf{multiplicity: 0..1} \newline
      timestamp of last modification of this entity; the timestamp is intended to be applied or updated when the entity is stored (e.g. in a database)

    \subsubsection{Entity.metaChecksum}
      \textbf{vodml-id: Entity.metaChecksum} \newline
      \textbf{type: \hyperref[sect:ivoa]{ivoa:anyURI}} \newline
      \textbf{multiplicity: 0..1} \newline
      checksum of this entity; (NEW in CAOM-2.3) The URI must conform to the pattern {algorithm}:{value}, for example: md5:4be91751541fd804e7207663a0822f56. The checksum of an entity is computed by accumulating byte representation of individual metadata values in the following order: (1) CaomEntity.id for entities, (2) CaomEntity.metaProducer, (3) state fields in alphabetic order (foo.a comes before foo.b) and using depth-first recursion (foo.abc.x comes before foo.def). Null values are ignored so that the addition of new fields in future versions will not change/invalidate existing checksums until values are assigned. Non-null values are converted to bytes as follows. TODO: add digestFieldNames to algorithm string: UTF-8 encoded bytes URI: UTF-8 encoded bytes of string representation float: IEEE754 single (4 bytes) double: IEEE754 double (8 bytes) boolean: convert to single byte, false=0, true=1 (1 bytes) byte: as-is (1 byte) short: (2 bytes, network byte order == big endian)) integer: (4 bytes, network byte order == big endian) long: (8 bytes, network byte order == big endian) date: truncate time to whole number of seconds and treat as a long (seconds since 1970-01-01 00:00:00 UTC)

    \subsubsection{Entity.metaProducer}
      \textbf{vodml-id: Entity.metaProducer} \newline
      \textbf{type: \hyperref[sect:ivoa]{ivoa:anyURI}} \newline
      \textbf{multiplicity: 0..1} \newline
      identifier for the producer of this entity and child entities with null metaProducer; (NEW in CAOM-2.4) The URI should conform to the pattern {organisation}:{software name-version} (for example: cadc:cfht2caom2-1.1) and identifies the tools used to produce the metadata. This information is intended for use by operators to help diagnose metadata issues.

  \subsection{Environment}
  \label{sect:Environment}
    collection of measured quantities that characterise the environment at the time of observation

    \subsubsection{Environment.name}
      \textbf{vodml-id: Environment.seeing} \newline
      \textbf{type: \hyperref[sect:ivoa]{ivoa:real}} \newline
      \textbf{multiplicity: 0..1} \newline
      typical atmospheric distortion (full-width-half-max of a point source)

    \subsubsection{Environment.humidity}
      \textbf{vodml-id: Environment.humidity} \newline
      \textbf{type: \hyperref[sect:ivoa]{ivoa:real}} \newline
      \textbf{multiplicity: 0..1} \newline
      fractional relative humidity [0,1]

    \subsubsection{Environment.elevation}
      \textbf{vodml-id: Environment.elevation} \newline
      \textbf{type: \hyperref[sect:ivoa]{ivoa:real}} \newline
      \textbf{multiplicity: 0..1} \newline
      angular elevation above horizon [0,90]

    \subsubsection{Environment.tau}
      \textbf{vodml-id: Environment.tau} \newline
      \textbf{type: \hyperref[sect:ivoa]{ivoa:real}} \newline
      \textbf{multiplicity: 0..1} \newline
      the opacity of the atmosphere [0,1]

    \subsubsection{Environment.wavelengthTau}
      \textbf{vodml-id: Environment.wavelengthTau} \newline
      \textbf{type: \hyperref[sect:ivoa]{ivoa:real}} \newline
      \textbf{multiplicity: 0..1} \newline
      wavelength at which opacity was measured

    \subsubsection{Environment.ambientTemp}
      \textbf{vodml-id: Environment.ambientTemp} \newline
      \textbf{type: \hyperref[sect:ivoa]{ivoa:real}} \newline
      \textbf{multiplicity: 0..1} \newline
      ambient temperature at the telescope

    \subsubsection{Environment.photometric}
      \textbf{vodml-id: Environment.photometric} \newline
      \textbf{type: \hyperref[sect:ivoa]{ivoa:boolean}} \newline
      \textbf{multiplicity: 0..1} \newline
      indicator that flux and/or color calibration is stable

  \subsection{Instrument}
  \label{sect:Instrument}
    the instrument used to acquire or create the observation; this could be used for both physical instruments that acquire data or software that generates it (e.g. simulated data)

    \subsubsection{Instrument.name}
      \textbf{vodml-id: Instrument.name} \newline
      \textbf{type: \hyperref[sect:ivoa]{ivoa:string}} \newline
      \textbf{multiplicity: 1} \newline
      common name for the instrument

    \subsubsection{Instrument.keywords}
      \textbf{vodml-id: Instrument.keywords} \newline
      \textbf{type: \hyperref[sect:ivoa]{ivoa:string}} \newline
      \textbf{multiplicity: 0..*} \newline
      additional keywords that describe the instrument or instrument configuration at the time of observation; keywords cannot contain the pipe (|) character - it is reserved for use in persistence systems (e.g. to store all keywords in a single column in a table)

  \subsection{Metrics}
  \label{sect:Metrics}
    collection of measured quantities that describe the content of the data

    \subsubsection{Metrics.sourceNumberDensity}
      \textbf{vodml-id: Metrics.sourceNumberDensity} \newline
      \textbf{type: \hyperref[sect:ivoa]{ivoa:real}} \newline
      \textbf{multiplicity: 0..1} \newline
      number of sources detected per unit area

    \subsubsection{Metrics.background}
      \textbf{vodml-id: Metrics.background} \newline
      \textbf{type: \hyperref[sect:ivoa]{ivoa:real}} \newline
      \textbf{multiplicity: 0..1} \newline
      background level

    \subsubsection{Metrics.backgroundStddev}
      \textbf{vodml-id: Metrics.backgroundStddev} \newline
      \textbf{type: \hyperref[sect:ivoa]{ivoa:real}} \newline
      \textbf{multiplicity: 0..1} \newline
      standard deviation in the background level

    \subsubsection{Metrics.fluxDensityLimit}
      \textbf{vodml-id: Metrics.fluxDensityLimit} \newline
      \textbf{type: \hyperref[sect:ivoa]{ivoa:real}} \newline
      \textbf{multiplicity: 0..1} \newline
      flux density with a signal:noise ratio of 10

    \subsubsection{Metrics.magLimit}
      \textbf{vodml-id: Metrics.magLimit} \newline
      \textbf{type: \hyperref[sect:ivoa]{ivoa:real}} \newline
      \textbf{multiplicity: 0..1} \newline
      magnitude with a signal:noise ratio of 10

    \subsubsection{Metrics.sampleSNR}
      \textbf{vodml-id: Metrics.sampleSNR} \newline
      \textbf{type: \hyperref[sect:ivoa]{ivoa:real}} \newline
      \textbf{multiplicity: 0..1} \newline
      signal:noise ratio for a representative subset of samples (new in CAOM-2.4)

  \subsection{Observable}
  \label{sect:Observable}
    description of the sample (pixel) values

    \subsubsection{Observable.ucd}
      \textbf{vodml-id: Observable.ucd} \newline
      \textbf{type: \hyperref[sect:UCD]{caom2:UCD}} \newline
      \textbf{multiplicity: 1} \newline
      Unified Content Descriptor (UCD) that says what kind of quantity is stored

    \subsubsection{Observable.calibration}
      \textbf{vodml-id: Observable.calibration} \newline
      \textbf{type: \hyperref[sect:CalibrationStatus]{caom2:CalibrationStatus}} \newline
      \textbf{multiplicity: 0..1} \newline
      term describing the calibration of the observable values

  \subsection{Observation (Abstract)}
  \label{sect:Observation}
    an observation is a single top-level entry in an astronomy data centre

    \subsubsection{Observation.collection}
      \textbf{vodml-id: Observation.collection} \newline
      \textbf{type: \hyperref[sect:ivoa]{ivoa:string}} \newline
      \textbf{multiplicity: 1} \newline
      the name of the data collection this observation belongs to

    \subsubsection{Observation.uri}
      \textbf{vodml-id: Observation.uri} \newline
      \textbf{type: \hyperref[sect:ivoa]{ivoa:anyURI}} \newline
      \textbf{multiplicity: 1} \newline
      unique logical identifier for this observation (NEW in CAOM-2.5)

    \subsubsection{Observation.metaRelease}
      \textbf{vodml-id: Observation.metaRelease} \newline
      \textbf{type: \hyperref[sect:ivoa]{ivoa:datetime}} \newline
      \textbf{multiplicity: 0..1} \newline
      timestamp after which metadata for the observation instance is public

    \subsubsection{Observation.sequenceNumber}
      \textbf{vodml-id: Observation.sequenceNumber} \newline
      \textbf{type: \hyperref[sect:ivoa]{ivoa:integer}} \newline
      \textbf{multiplicity: 0..1} \newline
      a collection-specific sequence number for observations; re-use or reset is collection specific

    \subsubsection{Observation.type}
      \textbf{vodml-id: Observation.type} \newline
      \textbf{type: \hyperref[sect:ivoa]{ivoa:string}} \newline
      \textbf{multiplicity: 0..1} \newline
      the type of observation (FITS OBSTYPE keyword); usually OBJECT for intent = science

    \subsubsection{Observation.intent}
      \textbf{vodml-id: Observation.intent} \newline
      \textbf{type: \hyperref[sect:ObservationIntentType]{caom2:ObservationIntentType}} \newline
      \textbf{multiplicity: 1} \newline
      the intent of the original observer in acquiring this data

    \subsubsection{Observation.metaReadGroups}
      \textbf{vodml-id: Observation.metaReadGroups} \newline
      \textbf{type: \hyperref[sect:ivoa]{ivoa:anyURI}} \newline
      \textbf{multiplicity: 0..*} \newline
      set of groups with read permission on observation metadata (new in 2.4)

    \subsubsection{Observation.algorithm}
      \textbf{vodml-id: Observation.algorithm} \newline
      \textbf{type: \hyperref[sect:Algorithm]{caom2:Algorithm}} \newline
      \textbf{multiplicity: 1} \newline
      the algorithm or process that created this observation

    \subsubsection{Observation.telescope}
      \textbf{vodml-id: Observation.telescope} \newline
      \textbf{type: \hyperref[sect:Telescope]{caom2:Telescope}} \newline
      \textbf{multiplicity: 0..1} \newline
      the telescope or facility where this observation was created

    \subsubsection{Observation.instrument}
      \textbf{vodml-id: Observation.instrument} \newline
      \textbf{type: \hyperref[sect:Instrument]{caom2:Instrument}} \newline
      \textbf{multiplicity: 0..1} \newline
      the instrument or detector used to acquire the data

    \subsubsection{Observation.environment}
      \textbf{vodml-id: Observation.environment} \newline
      \textbf{type: \hyperref[sect:Environment]{caom2:Environment}} \newline
      \textbf{multiplicity: 0..1} \newline
      the environmental conditions at the time of observation

    \subsubsection{Observation.proposal}
      \textbf{vodml-id: Observation.Proposal} \newline
      \textbf{type: \hyperref[sect:Proposal]{caom2:Proposal}} \newline
      \textbf{multiplicity: 0..1} \newline
      the science proposal underwhich this observation was created

    \subsubsection{Observation.target}
      \textbf{vodml-id: Observation.target} \newline
      \textbf{type: \hyperref[sect:Target]{caom2:Target}} \newline
      \textbf{multiplicity: 0..1} \newline
      the intended target of the observation

    \subsubsection{Observation.targetPosition}
      \textbf{vodml-id: Observation.targetPosition} \newline
      \textbf{type: \hyperref[sect:TargetPosition]{caom2:TargetPosition}} \newline
      \textbf{multiplicity: 0..1} \newline
      the intended target position for this observation

    \subsubsection{Observation.requirements}
      \textbf{vodml-id: Observation.requirements} \newline
      \textbf{type: \hyperref[sect:Requirements]{caom2:Requirements}} \newline
      \textbf{multiplicity: 0..1} \newline
      the observational requirements specified by the observer or proposal

    \subsubsection{Observation.planes}
      \textbf{vodml-id: Observation.planes} \newline
      \textbf{type: \hyperref[sect:Plane]{caom2:Plane}} \newline
      \textbf{multiplicity: 0..*} \newline
      the component planes belonging to this observation

  \subsection{Part}
  \label{sect:Part}
    format-specific name of this part; this is typically something like a FITS extension or a file within a container

    \subsubsection{Part.name}
      \textbf{vodml-id: Part.name} \newline
      \textbf{type: \hyperref[sect:ivoa]{ivoa:string}} \newline
      \textbf{multiplicity: 1} \newline
      the name of this part of the artifact; this is typically something like a FITS extension name or number or a filename

    \subsubsection{Part.productType}
      \textbf{vodml-id: Part.productType} \newline
      \textbf{type: \hyperref[sect:DataLinkSemantics]{caom2:DataLinkSemantics}} \newline
      \textbf{multiplicity: 0..1} \newline
      the relationship of this Artifact to the parent Plane (DataLink semantics)

    \subsubsection{Part.chunks}
      \textbf{vodml-id: Part.chunks} \newline
      \textbf{type: \hyperref[sect:Chunk]{caom2:Chunk}} \newline
      \textbf{multiplicity: 0..*} \newline
      component chunks that belong to this part

  \subsection{Plane}
  \label{sect:Plane}
    a component of an observation that describes one product of the observation

    \subsubsection{Plane.uri}
      \textbf{vodml-id: Plane.uri} \newline
      \textbf{type: \hyperref[sect:ivoa]{ivoa:anyURI}} \newline
      \textbf{multiplicity: 0..1} \newline
      unique logical identifier for this product; typically made by adding additional path component(s) to the Observation.uri (NEW in CAOM-2.5)

    \subsubsection{Plane.metaRelease}
      \textbf{vodml-id: Plane.metaRelease} \newline
      \textbf{type: \hyperref[sect:ivoa]{ivoa:datetime}} \newline
      \textbf{multiplicity: 0..1} \newline
      timestamp after which metadata for the plane is public; this metaRelease timestamp applies to all children of the plane and to artifacts with releaseType=meta

    \subsubsection{Plane.metaReadGroups}
      \textbf{vodml-id: Plane.metaReadGroups} \newline
      \textbf{type: \hyperref[sect:ivoa]{ivoa:anyURI}} \newline
      \textbf{multiplicity: 0..*} \newline
      list of groups (of users) that are allowed to view the metadata of the plane; this is applicable when metaRelease is null or in the future (new in CAOM-2.4)

    \subsubsection{Plane.dataRelease}
      \textbf{vodml-id: Plane.dataRelease} \newline
      \textbf{type: \hyperref[sect:ivoa]{ivoa:datetime}} \newline
      \textbf{multiplicity: 0..1} \newline
      timestamp after which data for the plane is public; this dataRelease timestamp applies to all children of the plane and to artifacts with releaseType=data

    \subsubsection{Plane.dataReadGroups}
      \textbf{vodml-id: Plane.dataReadGroups} \newline
      \textbf{type: \hyperref[sect:ivoa]{ivoa:anyURI}} \newline
      \textbf{multiplicity: 0..*} \newline
      list of groups (of users) that are allowed to access the data of the plane; this is applicable when dataRelease is null or in the future (new in CAOM-2.4)

    \subsubsection{Plane.calibrationLevel}
      \textbf{vodml-id: Plane.calibrationLevel} \newline
      \textbf{type: \hyperref[sect:CalibrationLevel]{caom2:CalibrationLevel}} \newline
      \textbf{multiplicity: 0..1} \newline
      standard classification of the degree to which the data is calibrated

    \subsubsection{Plane.dataProductType}
      \textbf{vodml-id: Plane.dataProductType} \newline
      \textbf{type: \hyperref[sect:DataProductType]{caom2:DataProductType}} \newline
      \textbf{multiplicity: 0..1} \newline
      standard classification of the type of data product; describes the logical data type for the main artifacts

    \subsubsection{Plane.observable}
      \textbf{vodml-id: Plane.observable} \newline
      \textbf{type: \hyperref[sect:Observable]{caom2:Observable}} \newline
      \textbf{multiplicity: 0..1} \newline
      description of the sample (pixel) values; (new in CAOM-2.4) In previous versions the observable was assumed to be flux or intensity of EM radiation.

    \subsubsection{Plane.quality}
      \textbf{vodml-id: Plane.quality} \newline
      \textbf{type: \hyperref[sect:DataQuality]{caom2:DataQuality}} \newline
      \textbf{multiplicity: 0..1} \newline
      flag indicating the quality of the data

    \subsubsection{Plane.metrics}
      \textbf{vodml-id: Plane.metrics} \newline
      \textbf{type: \hyperref[sect:Metrics]{caom2:Metrics}} \newline
      \textbf{multiplicity: 0..1} \newline
      collection of measured quantities that describe the content of the data

    \subsubsection{Plane.position}
      \textbf{vodml-id: Plane.position} \newline
      \textbf{type: \hyperref[sect:Position]{caom2:Position}} \newline
      \textbf{multiplicity: 0..1} \newline
      description of the position(s) included in the data

    \subsubsection{Plane.energy}
      \textbf{vodml-id: Plane.energy} \newline
      \textbf{type: \hyperref[sect:Energy]{caom2:Energy}} \newline
      \textbf{multiplicity: 0..1} \newline
      descritpion of the energy(ies) included in the data

    \subsubsection{Plane.time}
      \textbf{vodml-id: Plane.time} \newline
      \textbf{type: \hyperref[sect:Time]{caom2:Time}} \newline
      \textbf{multiplicity: 0..1} \newline
      description of the time(s) included in the data

    \subsubsection{Plane.polarization}
      \textbf{vodml-id: Plane.polarization} \newline
      \textbf{type: \hyperref[sect:Polarization]{caom2:Polarization}} \newline
      \textbf{multiplicity: 0..1} \newline
      description of the polarization(s) included in the data

    \subsubsection{Plane.custom}
      \textbf{vodml-id: Plane.custom} \newline
      \textbf{type: \hyperref[sect:CustomAxis]{caom2:CustomAxis}} \newline
      \textbf{multiplicity: 0..1} \newline
      description of a custom coordinate axis in the data (new in CAOM-2.4); Since different custom coordinate types can be used with different planes, instances of CustomAxis can only be compared sensibly if they have the same coordinate type.

    \subsubsection{Plane.uv}
      \textbf{vodml-id: Plane.uv} \newline
      \textbf{type: \hyperref[sect:Visibility]{caom2:Visibility}} \newline
      \textbf{multiplicity: 0..1} \newline
      [TODO add description!]

    \subsubsection{Plane.provenance}
      \textbf{vodml-id: Plane.provenance} \newline
      \textbf{type: \hyperref[sect:Provenance]{caom2:Provenance}} \newline
      \textbf{multiplicity: 0..1} \newline
      description of the provenance of the data

    \subsubsection{Plane.artifacts}
      \textbf{vodml-id: Plane.artifacts} \newline
      \textbf{type: \hyperref[sect:Artifact]{caom2:Artifact}} \newline
      \textbf{multiplicity: 0..*} \newline
      the component artifacts belonging to this plane

  \subsection{Polarization}
  \label{sect:Polarization}
    description of polarization measurements included in the data

    \subsubsection{Polarization.states}
      \textbf{vodml-id: Polarization.states} \newline
      \textbf{type: \hyperref[sect:PolarizationState]{caom2:PolarizationState}} \newline
      \textbf{multiplicity: 1..*} \newline
      standard polarization states included (cardinality changed in CAOM-2.5)

    \subsubsection{Polarization.dimension}
      \textbf{vodml-id: Polarization.dimension} \newline
      \textbf{type: \hyperref[sect:ivoa]{ivoa:integer}} \newline
      \textbf{multiplicity: 0..1} \newline
      number of polarization states included

  \subsection{PolarizationState}
  \label{sect:PolarizationState}
    this class defines constants for the PolarizationState vocabulary (CHANGED from enumeration in CAOM-2.5)

  \subsection{Position}
  \label{sect:Position}
    description of the position coverage and sampling of the data

    \subsubsection{Position.bounds}
      \textbf{vodml-id: Position.bounds} \newline
      \textbf{type: \hyperref[sect:types.Shape]{caom2:types.Shape}} \newline
      \textbf{multiplicity: 1} \newline
      spatial boundary that includes the data (cardinality changed in CAOM-2.5)

    \subsubsection{Position.samples}
      \textbf{vodml-id: Position.samples} \newline
      \textbf{type: \hyperref[sect:types.MultiShape]{caom2:types.MultiShape}} \newline
      \textbf{multiplicity: 1} \newline
      detailed sub-samples of the bounds (refactored in CAOM-2.5)

    \subsubsection{Position.minBounds}
      \textbf{vodml-id: Position.minBounds} \newline
      \textbf{type: \hyperref[sect:types.Shape]{caom2:types.Shape}} \newline
      \textbf{multiplicity: 0..1} \newline
      minimum spatial boundary that includes the data; this value is smaller than the bounds (maximum) when the field-of-view varies because it is dependent on another axis (usually energy) (new in CAOM-2.5)

    \subsubsection{Position.dimension}
      \textbf{vodml-id: Position.dimension} \newline
      \textbf{type: \hyperref[sect:Dimension2D]{caom2:Dimension2D}} \newline
      \textbf{multiplicity: 0..1} \newline
      number of separate measurements (pixels) along each axis

    \subsubsection{Position.maxAngularScale}
      \textbf{vodml-id: Position.maxAngularScale} \newline
      \textbf{type: \hyperref[sect:types.Interval]{caom2:types.Interval}} \newline
      \textbf{multiplicity: 0..1} \newline
      maximum size of spatial structure (signal) that can be recovered or seen in the data (new in CAOM-2.5)

    \subsubsection{Position.resolution}
      \textbf{vodml-id: Position.resolution} \newline
      \textbf{type: \hyperref[sect:ivoa]{ivoa:real}} \newline
      \textbf{multiplicity: 0..1} \newline
      mean spatial resolution (full-width-half-max) per pixel

    \subsubsection{Position.resolutionBounds}
      \textbf{vodml-id: Position.resolutionBounds} \newline
      \textbf{type: \hyperref[sect:types.Interval]{caom2:types.Interval}} \newline
      \textbf{multiplicity: 0..1} \newline
      range of resolution within the bounds

    \subsubsection{Position.sampleSize}
      \textbf{vodml-id: Position.sampleSize} \newline
      \textbf{type: \hyperref[sect:ivoa]{ivoa:real}} \newline
      \textbf{multiplicity: 0..1} \newline
      median pixel size

    \subsubsection{Position.calibration}
      \textbf{vodml-id: Position.calibration} \newline
      \textbf{type: \hyperref[sect:CalibrationStatus]{caom2:CalibrationStatus}} \newline
      \textbf{multiplicity: 0..1} \newline
      term describing the calibration of the position axis values new in CAOM-2.5)

  \subsection{Proposal}
  \label{sect:Proposal}
    description of the science proposal or programme that initiated the observation

    \subsubsection{Proposal.proposalID}
      \textbf{vodml-id: Proposal.proposalID} \newline
      \textbf{type: \hyperref[sect:ivoa]{ivoa:string}} \newline
      \textbf{multiplicity: 1} \newline
      collection-specific identifier for the proposal

    \subsubsection{Proposal.pi}
      \textbf{vodml-id: Proposal.pi} \newline
      \textbf{type: \hyperref[sect:ivoa]{ivoa:string}} \newline
      \textbf{multiplicity: 0..1} \newline
      proper name of the principal investigator

    \subsubsection{Proposal.project}
      \textbf{vodml-id: Proposal.project} \newline
      \textbf{type: \hyperref[sect:ivoa]{ivoa:string}} \newline
      \textbf{multiplicity: 0..1} \newline
      common name of the project this proposal belongs to; typically used for larger or long-running projects that include mutliple proposals

    \subsubsection{Proposal.title}
      \textbf{vodml-id: Proposal.title} \newline
      \textbf{type: \hyperref[sect:ivoa]{ivoa:string}} \newline
      \textbf{multiplicity: 0..1} \newline
      title of the proposal

    \subsubsection{Proposal.keywords}
      \textbf{vodml-id: Proposal.keywords} \newline
      \textbf{type: \hyperref[sect:ivoa]{ivoa:string}} \newline
      \textbf{multiplicity: 0..*} \newline
      additional keywords that describe the science goals of the proposal; keywords cannot contain the pipe (|) character - it is reserved for use in persistence systems (e.g. to store all keywords in a single column in a table)

    \subsubsection{Proposal.reference}
      \textbf{vodml-id: Proposal.reference} \newline
      \textbf{type: \hyperref[sect:ivoa]{ivoa:anyURI}} \newline
      \textbf{multiplicity: 0..1} \newline
      identifier for external resource with proposal details

  \subsection{Provenance}
  \label{sect:Provenance}
    description of how this data was produced

    \subsubsection{Provenance.name}
      \textbf{vodml-id: Provenance.name} \newline
      \textbf{type: \hyperref[sect:ivoa]{ivoa:string}} \newline
      \textbf{multiplicity: 1} \newline
      collection-specific common name of the process

    \subsubsection{Provenance.reference}
      \textbf{vodml-id: Provenance.reference} \newline
      \textbf{type: \hyperref[sect:ivoa]{ivoa:anyURI}} \newline
      \textbf{multiplicity: 0..1} \newline
      identifier for external resource with proposal details

    \subsubsection{Provenance.version}
      \textbf{vodml-id: Provenance.version} \newline
      \textbf{type: \hyperref[sect:ivoa]{ivoa:string}} \newline
      \textbf{multiplicity: 0..1} \newline
      version of the software or process that produced the data

    \subsubsection{Provenance.project}
      \textbf{vodml-id: Provenance.project} \newline
      \textbf{type: \hyperref[sect:ivoa]{ivoa:string}} \newline
      \textbf{multiplicity: 0..1} \newline
      name of the project that produced the data; data produced in a uniform way are typically labelled with the same project name

    \subsubsection{Provenance.producer}
      \textbf{vodml-id: Provenance.producer} \newline
      \textbf{type: \hyperref[sect:ivoa]{ivoa:string}} \newline
      \textbf{multiplicity: 0..1} \newline
      common name of the entity (person, institute, etc) responsible for producing the data

    \subsubsection{Provenance.runID}
      \textbf{vodml-id: Provenance.runID} \newline
      \textbf{type: \hyperref[sect:ivoa]{ivoa:string}} \newline
      \textbf{multiplicity: 0..1} \newline
      collection-specific identifier for the processing instance that produced the data; this identifier can typcially be traced in log files or logging systems

    \subsubsection{Provenance.lastExecuted}
      \textbf{vodml-id: Provenance.lastExecuted} \newline
      \textbf{type: \hyperref[sect:ivoa]{ivoa:string}} \newline
      \textbf{multiplicity: 0..1} \newline
      timestamp describing when this process last ran and produced data

    \subsubsection{Provenance.keywords}
      \textbf{vodml-id: Provenance.keywords} \newline
      \textbf{type: \hyperref[sect:ivoa]{ivoa:string}} \newline
      \textbf{multiplicity: 0..*} \newline
      additional keywords that describe the processing; this may include both general descriptive words and those specific to this particular execution of the processing; keywords cannot contain the pipe (|) character - it is reserved for use in persistence systems (e.g. to store all keywords in a single column in a table)

    \subsubsection{Provenance.inputs}
      \textbf{vodml-id: Provenance.inputs} \newline
      \textbf{type: \hyperref[sect:ivoa]{ivoa:anyURI}} \newline
      \textbf{multiplicity: 0..*} \newline
      local identifier for input planes; these are the actual inputs that went into the product

  \subsection{Quality}
  \label{sect:Quality}
    vocabulary term used in CAOM; this class defines constants for the CAOM Quality vocabulary (CHANGED from enumeration in CAOM-2.4)

  \subsection{Requirements}
  \label{sect:Requirements}
    the observational requirements specified by the proposal

    \subsubsection{Requirements.flag}
      \textbf{vodml-id: Requirements.flag} \newline
      \textbf{type: \hyperref[sect:Status]{caom2:Status}} \newline
      \textbf{multiplicity: 1} \newline
      flag indicating degree to which requirements were satisfied by the observation

  \subsection{SimpleObservation}
  \label{sect:SimpleObservation}
    an observation created directly by operating an instrument or process

  \subsection{Status}
  \label{sect:Status}
    vocabulary term used in CAOM; this class defines constants for the CAOM Status vocabulary (CHANGED from enumeration in CAOM-2.4)

  \subsection{Target}
  \label{sect:Target}
    the target of an observation

    \subsubsection{Target.name}
      \textbf{vodml-id: Target.name} \newline
      \textbf{type: \hyperref[sect:ivoa]{ivoa:string}} \newline
      \textbf{multiplicity: 1} \newline
      proper name of the target

    \subsubsection{Target.targetID}
      \textbf{vodml-id: Target.targetID} \newline
      \textbf{type: \hyperref[sect:ivoa]{ivoa:anyURI}} \newline
      \textbf{multiplicity: 0..1} \newline
      resolvable target identifier (new in CAOM-2.4); the targetID URI should be of the form {scheme}:{id} so it can be resolved (for example: naif:170100)

    \subsubsection{Target.type}
      \textbf{vodml-id: Target.type} \newline
      \textbf{type: \hyperref[sect:TargetType]{caom2:TargetType}} \newline
      \textbf{multiplicity: 0..1} \newline
      type of target; typically used to figure out what the target name means and where to look for additional information about it

    \subsubsection{Target.redshift}
      \textbf{vodml-id: Target.redshift} \newline
      \textbf{type: \hyperref[sect:ivoa]{ivoa:real}} \newline
      \textbf{multiplicity: 0..1} \newline
      cosmological redshift of the target

    \subsubsection{Target.standard}
      \textbf{vodml-id: Target.standard} \newline
      \textbf{type: \hyperref[sect:ivoa]{ivoa:boolean}} \newline
      \textbf{multiplicity: 0..1} \newline
      indicates that the target is typically used as a standard (astrometric, photometric, etc)

    \subsubsection{Target.moving}
      \textbf{vodml-id: Target.moving} \newline
      \textbf{type: \hyperref[sect:ivoa]{ivoa:boolean}} \newline
      \textbf{multiplicity: 0..1} \newline
      indicates that the target is a moving object; used for solar system objects but not high proper motion stars

    \subsubsection{Target.keywords}
      \textbf{vodml-id: Target.keywords} \newline
      \textbf{type: \hyperref[sect:ivoa]{ivoa:string}} \newline
      \textbf{multiplicity: 0..*} \newline
      additional keywords that describe the target; keywords cannot contain the pipe (|) character - it is reserved for use in persistence systems (e.g. to store all keywords in a single column in a table)

  \subsection{TargetPosition}
  \label{sect:TargetPosition}
    the intended position of the observation (not the position of the intended or actual target)

    \subsubsection{TargetPosition.coordsys}
      \textbf{vodml-id: TargetPosition.coordsys} \newline
      \textbf{type: \hyperref[sect:ivoa]{ivoa:string}} \newline
      \textbf{multiplicity: 1} \newline
      the coordinate system of the coordinates

    \subsubsection{TargetPosition.equinox}
      \textbf{vodml-id: TargetPosition.equinox} \newline
      \textbf{type: \hyperref[sect:ivoa]{ivoa:real}} \newline
      \textbf{multiplicity: 0..1} \newline
      the equinox of the coordinates

    \subsubsection{TargetPosition.coordinates}
      \textbf{vodml-id: TargetPosition.coordinates} \newline
      \textbf{type: \hyperref[sect:types.Point]{caom2:types.Point}} \newline
      \textbf{multiplicity: 1} \newline
      the coordinates

  \subsection{TargetType}
  \label{sect:TargetType}
    vocabulary term used in CAOM; this class defines constants for the CAOM TargetType vocabulary (CHANGED from enumeration in CAOM-2.4)

  \subsection{Telescope}
  \label{sect:Telescope}
    the telescope used to acquire the data for an observation

    \subsubsection{Telescope.name}
      \textbf{vodml-id: Telescope.name} \newline
      \textbf{type: \hyperref[sect:ivoa]{ivoa:string}} \newline
      \textbf{multiplicity: 1} \newline
      common name of the telescope; TBD: reference to a standard list of names?

    \subsubsection{Telescope.geoLocationX}
      \textbf{vodml-id: Telescope.geoLocationX} \newline
      \textbf{type: \hyperref[sect:ivoa]{ivoa:real}} \newline
      \textbf{multiplicity: 0..1} \newline
      x-coordinate of the geocentric location of the telescope at the time of observation (see FITS WCS Paper III)

    \subsubsection{Telescope.geoLocationY}
      \textbf{vodml-id: Telescope.geoLocationY} \newline
      \textbf{type: \hyperref[sect:ivoa]{ivoa:real}} \newline
      \textbf{multiplicity: 0..1} \newline
      y-coordinate of the geocentric location of the telescope at the time of observation (see FITS WCS Paper III)

    \subsubsection{Telescope.geoLocationZ}
      \textbf{vodml-id: Telescope.geoLocationZ} \newline
      \textbf{type: \hyperref[sect:ivoa]{ivoa:real}} \newline
      \textbf{multiplicity: 0..1} \newline
      z-coordinate of the geocentric location of the telescope at the time of observation (see FITS WCS Paper III)

    \subsubsection{Telescope.keywords}
      \textbf{vodml-id: Telescope.keywords} \newline
      \textbf{type: \hyperref[sect:ivoa]{ivoa:string}} \newline
      \textbf{multiplicity: 0..*} \newline
      additional keywords that describe the telescope or telscope configuration at the time of observation; keywords cannot contain the pipe (|) character - it is reserved for use in persistence systems (e.g. to store all keywords in a single column in a table)

    \subsubsection{Telescope.trackingMode}
      \textbf{vodml-id: Telescope.trackingMode} \newline
      \textbf{type: \hyperref[sect:VocabularyTerm]{caom2:VocabularyTerm}} \newline
      \textbf{multiplicity: 0..1} \newline
      term from the (currently non-existent) IVOA Tracking vocabulary used to indicate how the telescope moves during data acquisition

  \subsection{Time}
  \label{sect:Time}
    description of the time coverage and sampling of the data

    \subsubsection{Time.bounds}
      \textbf{vodml-id: Time.bounds} \newline
      \textbf{type: \hyperref[sect:types.Interval]{caom2:types.Interval}} \newline
      \textbf{multiplicity: 1} \newline
      time bounds that include the data (cardinality changed in CAOM-2.5)

    \subsubsection{Time.samples}
      \textbf{vodml-id: Time.samples} \newline
      \textbf{type: \hyperref[sect:types.Interval]{caom2:types.Interval}} \newline
      \textbf{multiplicity: 1..*} \newline
      detailed time coverage that include the data (refactored in CAOM-2.5)

    \subsubsection{Time.calibration}
      \textbf{vodml-id: Time.calibration} \newline
      \textbf{type: \hyperref[sect:CalibrationStatus]{caom2:CalibrationStatus}} \newline
      \textbf{multiplicity: 0..1} \newline
      term describing the calibration of the time axis values

    \subsubsection{Time.dimension}
      \textbf{vodml-id: Time.dimension} \newline
      \textbf{type: \hyperref[sect:ivoa]{ivoa:integer}} \newline
      \textbf{multiplicity: 0..1} \newline
      number of discrete samples (pixels) on the time axis

    \subsubsection{Time.resolution}
      \textbf{vodml-id: Time.resolution} \newline
      \textbf{type: \hyperref[sect:ivoa]{ivoa:real}} \newline
      \textbf{multiplicity: 0..1} \newline
      median temporal resolution per pixel

    \subsubsection{Time.resolutionBounds}
      \textbf{vodml-id: Time.resolutionBounds} \newline
      \textbf{type: \hyperref[sect:types.Interval]{caom2:types.Interval}} \newline
      \textbf{multiplicity: 0..1} \newline
      range of resolution within the bounds

    \subsubsection{Time.sampleSize}
      \textbf{vodml-id: Time.sampleSize} \newline
      \textbf{type: \hyperref[sect:ivoa]{ivoa:real}} \newline
      \textbf{multiplicity: 0..1} \newline
      median pixel size

    \subsubsection{Time.exposure}
      \textbf{vodml-id: Time.exposure} \newline
      \textbf{type: \hyperref[sect:ivoa]{ivoa:real}} \newline
      \textbf{multiplicity: 0..1} \newline
      mean exposure time per pixel

    \subsubsection{Time.exposureBounds}
      \textbf{vodml-id: Time.exposureBounds} \newline
      \textbf{type: \hyperref[sect:types.Interval]{caom2:types.Interval}} \newline
      \textbf{multiplicity: 0..1} \newline
      range of exposure within the bounds

  \subsection{Tracking}
  \label{sect:Tracking}
    term from the Tracking vocabulary NEW from enumeration in CAOM-2.5)

  \subsection{UCD}
  \label{sect:UCD}
    term from the UCD1+ vocabulary (NEW from enumeration in CAOM-2.5)

  \subsection{Visibility}
  \label{sect:Visibility}
    description of a UV-plane for interferometry data (new in CAOM-2.5)

    \subsubsection{Visibility.distance}
      \textbf{vodml-id: Visibility.distance} \newline
      \textbf{type: \hyperref[sect:types.Interval]{caom2:types.Interval}} \newline
      \textbf{multiplicity: 1} \newline
      range of distances in the UV plane

    \subsubsection{Visibility.distance}
      \textbf{vodml-id: Visibility.distributionEccentricity} \newline
      \textbf{type: \hyperref[sect:ivoa]{ivoa:real}} \newline
      \textbf{multiplicity: 1} \newline
      eccentricity of the distribtuion of ??? in [0,1]

    \subsubsection{Visibility.distance}
      \textbf{vodml-id: Visibility.distributionFill} \newline
      \textbf{type: \hyperref[sect:ivoa]{ivoa:real}} \newline
      \textbf{multiplicity: 1} \newline
      fill-factor of the distribtuion of ??? in [0,1]

  \subsection{VocabularyTerm (Abstract)}
  \label{sect:VocabularyTerm}
    base class of a single term (word) in a vocabulary (NEW in CAOM-2.3)

    \subsubsection{VocabularyTerm.namespace}
      \textbf{vodml-id: VocabularyTerm.namespace} \newline
      \textbf{type: \hyperref[sect:ivoa]{ivoa:anyURI}} \newline
      \textbf{multiplicity: 1} \newline
      globally unique namespace for the vocabulary

    \subsubsection{VocabularyTerm.term}
      \textbf{vodml-id: VocabularyTerm.term} \newline
      \textbf{type: \hyperref[sect:ivoa]{ivoa:string}} \newline
      \textbf{multiplicity: 1} \newline
      the word from the vocabulary

    \subsubsection{VocabularyTerm.term}
      \textbf{vodml-id: VocabularyTerm.base} \newline
      \textbf{type: \hyperref[sect:ivoa]{ivoa:boolean}} \newline
      \textbf{multiplicity: 1} \newline
      flag indicating of the vocabulary namespace is a base vocabulary

  \subsection{uuid}
  \label{sect:uuid}
  represents a 128-bit binary ID in the canonical ascii UUID format

  \subsection{ObservationIntentType}
  \label{sect:ObservationIntentType}

  the intent of the original creator (usually observer) in acquiring this observation

  \noindent \underline{Enumeration Literals}
  \vspace{-\parsep}
  \small
  \begin{itemize}
  
    \item[\textbf{science}]: \textbf{vodml-id:} ObservationIntentType.SCIENCE \newline
          \textbf{description:} the intent of this observation was to create science data
    \item[\textbf{calibration}]: \textbf{vodml-id:} ObservationIntentType.CALIBRATION \newline
          \textbf{description:} the intent of this observation was to create calibration data
    \item[\textbf{calibration}]: \textbf{vodml-id:} ObservationIntentType.OUTREACH \newline
          \textbf{description:} the intent of this observation was to create public outreach content
  \end{itemize}
  \normalsize


  \subsection{CalibrationLevel}
  \label{sect:CalibrationLevel}

  the degree to which data has been calibrated to remove instrumental effects; issue: there is no way to convey the integer serialised values here so it is in the description of each value

  \noindent \underline{Enumeration Literals}
  \vspace{-\parsep}
  \small
  \begin{itemize}
  
    \item[\textbf{PLANNED}]: \textbf{vodml-id:} CalibrationLevel.PLANNED \newline
          \textbf{description:} (-1) planned data product that does not yet exist
    \item[\textbf{RAW\_INSTRUMENTAL}]: \textbf{vodml-id:} CalibrationLevel.RAW\_INSTRUMENTAL \newline
          \textbf{description:} (0) raw data in some opaque instrument-specific format
    \item[\textbf{RAW\_STANDARD}]: \textbf{vodml-id:} CalibrationLevel.RAW\_STANDARD \newline
          \textbf{description:} (1) raw data in a common format
    \item[\textbf{CALIBRATED}]: \textbf{vodml-id:} CalibrationLevel.CALIBRATED \newline
          \textbf{description:} (2) standard calibration steps have been applied
    \item[\textbf{PRODUCT}]: \textbf{vodml-id:} CalibrationLevel.PRODUCT \newline
          \textbf{description:} (3) additional non-standard calibration steps have been applied
    \item[\textbf{ANALYSIS\_PRODUCT}]: \textbf{vodml-id:} CalibrationLevel.ANALYSIS\_PRODUCT \newline
          \textbf{description:} (4) : data product from scientific analysis
  \end{itemize}
  \normalsize


  \subsection{EnergyBand}
  \label{sect:EnergyBand}

  a general set of energy regions that span the electromagnetic spectrum; work-around: using the name of the literal to convey the serialised value

  \noindent \underline{Enumeration Literals}
  \vspace{-\parsep}
  \small
  \begin{itemize}
  
    \item[\textbf{Radio}]: \textbf{vodml-id:} EnergyBand.RADIO \newline
          \textbf{description:} wavelength greater than ~10mm
    \item[\textbf{Millimeter}]: \textbf{vodml-id:} EnergyBand.MILLIMETER \newline
          \textbf{description:} wavelength from 0.1 to 10mm
    \item[\textbf{Infrared}]: \textbf{vodml-id:} EnergyBand.INFRARED \newline
          \textbf{description:} wavelength from 1um to 0.1mm
    \item[\textbf{Optical}]: \textbf{vodml-id:} EnergyBand.OPTICAL \newline
          \textbf{description:} wavelength from 300nm to 1um
    \item[\textbf{UV}]: \textbf{vodml-id:} EnergyBand.UV \newline
          \textbf{description:} wavelength from 100 to 300nm
    \item[\textbf{EUV}]: \textbf{vodml-id:} EnergyBand.EUV \newline
          \textbf{description:} wavelength from 10 to 100nm
    \item[\textbf{Xray}]: \textbf{vodml-id:} EnergyBand.XRAY \newline
          \textbf{description:} energy from 0.12 to 120keV
    \item[\textbf{Gammaray}]: \textbf{vodml-id:} EnergyBand.GAMMARAY \newline
          \textbf{description:} energy greater than ~120keV
  \end{itemize}
  \normalsize


  \subsection{ReleaseType}
  \label{sect:ReleaseType}

  a flag indicating how an artifact is classified to determine access permissions; work-around: using the name of the literal to convey the serialised value

  \noindent \underline{Enumeration Literals}
  \vspace{-\parsep}
  \small
  \begin{itemize}
  
    \item[\textbf{data}]: \textbf{vodml-id:} ReleaseType.DATA \newline
          \textbf{description:} access permission checks assume the protected item is data
    \item[\textbf{meta}]: \textbf{vodml-id:} ReleaseType.META \newline
          \textbf{description:} access permission checks assume the protected item is metadata
  \end{itemize}
  \normalsize


\pagebreak
\section{Package: types }

  % INSERT FIGURE HERE
  %\begin{figure}[h]
  %\begin{center}
  %  \includegraphics[width=\textwidth]{????.png}
  %  \caption{???}\label{fig:????}
  %\end{center}
  %\end{figure}

  data types

  \subsection{Circle}
  \label{sect:types.Circle}
    a circular region on the sky

    \subsubsection{Circle.center}
      \textbf{vodml-id: types.Circle.center} \newline
      \textbf{type: \hyperref[sect:types.Point]{caom2:types.Point}} \newline
      \textbf{multiplicity: 1} \newline
      [TODO add description!]

    \subsubsection{Circle.radius}
      \textbf{vodml-id: types.Circle.radius} \newline
      \textbf{type: \hyperref[sect:ivoa]{ivoa:real}} \newline
      \textbf{multiplicity: 1} \newline
      [TODO add description!]

  \subsection{Interval}
  \label{sect:types.Interval}
    a set of numeric values defined by a lower and upper bound (bounds included: [a,b])

    \subsubsection{Interval.lower}
      \textbf{vodml-id: types.Interval.lower} \newline
      \textbf{type: \hyperref[sect:ivoa]{ivoa:real}} \newline
      \textbf{multiplicity: 1} \newline
      [TODO add description!]

    \subsubsection{Interval.upper}
      \textbf{vodml-id: types.Interval.upper} \newline
      \textbf{type: \hyperref[sect:ivoa]{ivoa:real}} \newline
      \textbf{multiplicity: 1} \newline
      [TODO add description!]

  \subsection{MultiShape}
  \label{sect:types.MultiShape}
    multiple simple shapess describing disconnected regions as a single value

    \subsubsection{MultiShape.shapes}
      \textbf{vodml-id: MultiShape.shapes} \newline
      \textbf{type: \hyperref[sect:types.Shape]{caom2:types.Shape}} \newline
      \textbf{multiplicity: 1..*} \newline
      [TODO add description!]

  \subsection{Point}
  \label{sect:types.Point}
    location on the sky

    \subsubsection{Point.cval1}
      \textbf{vodml-id: Point.cval1} \newline
      \textbf{type: \hyperref[sect:ivoa]{ivoa:real}} \newline
      \textbf{multiplicity: 1} \newline
      [TODO add description!]

    \subsubsection{Point.cval1}
      \textbf{vodml-id: Point.cval2} \newline
      \textbf{type: \hyperref[sect:ivoa]{ivoa:real}} \newline
      \textbf{multiplicity: 1} \newline
      [TODO add description!]

  \subsection{Polygon}
  \label{sect:types.Polygon}
    a simple polygon region on the sky defined a sequence of points connected by great-circle segments

    \subsubsection{Polygon.points}
      \textbf{vodml-id: type.Polygon.points} \newline
      \textbf{type: \hyperref[sect:types.Point]{caom2:types.Point}} \newline
      \textbf{multiplicity: 3..*} \newline
      [TODO add description!]

  \subsection{Shape (Abstract)}
  \label{sect:types.Shape}
    [TODO add description!]

\pagebreak
\section{Package: wcs }

  % INSERT FIGURE HERE
  %\begin{figure}[h]
  %\begin{center}
  %  \includegraphics[width=\textwidth]{????.png}
  %  \caption{???}\label{fig:????}
  %\end{center}
  %\end{figure}

  World Coordinate System (WCS) data types

  \subsection{Axis}
  \label{sect:Axis}
    one-dimensional coordinate axis description

    \subsubsection{Axis.ctype}
      \textbf{vodml-id: Axis.ctype} \newline
      \textbf{type: \hyperref[sect:ivoa]{ivoa:string}} \newline
      \textbf{multiplicity: 1} \newline
      [TODO add description!]

    \subsubsection{Axis.ctype}
      \textbf{vodml-id: Axis.cunit} \newline
      \textbf{type: \hyperref[sect:ivoa]{ivoa:string}} \newline
      \textbf{multiplicity: 0..1} \newline
      [TODO add description!]

  \subsection{Coord2D}
  \label{sect:Coord2D}
    a two-dimensional (pair) of reference coordinates

    \subsubsection{Coord2D.coord1}
      \textbf{vodml-id: Coord2D.coord1} \newline
      \textbf{type: \hyperref[sect:RefCoord]{caom2:RefCoord}} \newline
      \textbf{multiplicity: 1} \newline
      [TODO add description!]

    \subsubsection{Coord2D.coord2}
      \textbf{vodml-id: Coord2D.coord2} \newline
      \textbf{type: \hyperref[sect:RefCoord]{caom2:RefCoord}} \newline
      \textbf{multiplicity: 1} \newline
      [TODO add description!]

  \subsection{CoordAxis1D}
  \label{sect:CoordAxis1D}
    a one-dimensional coordinate axis: quantity, values, errors; it is usually only necessary to specify one of the range, bounds, or function as they describe the world and pixel coordinate coverage at different levels of detail and the less detailed description is redundant (exception: when the range or bounds and function are both specified, the range/bounds is a subset of the pixels described by the function and denotes the valid pixels

    \subsubsection{CoordAxis1D.axis}
      \textbf{vodml-id: CoordAxis1D.axis} \newline
      \textbf{type: \hyperref[sect:Axis]{caom2:Axis}} \newline
      \textbf{multiplicity: 1} \newline
      description of the quantity

    \subsubsection{CoordAxis1D.error}
      \textbf{vodml-id: CoordAxis1D.error} \newline
      \textbf{type: \hyperref[sect:CoordError]{caom2:CoordError}} \newline
      \textbf{multiplicity: 0..1} \newline
      errors

    \subsubsection{CoordAxis1D.range}
      \textbf{vodml-id: CoordAxis1D.range} \newline
      \textbf{type: \hyperref[sect:CoordRange1D]{caom2:CoordRange1D}} \newline
      \textbf{multiplicity: 0..1} \newline
      pixel and world coordinate values covered by this axis (min,max)

    \subsubsection{CoordAxis1D.bounds}
      \textbf{vodml-id: CoordAxis1D.bounds} \newline
      \textbf{type: \hyperref[sect:CoordBounds1D]{caom2:CoordBounds1D}} \newline
      \textbf{multiplicity: 0..1} \newline
      pixel and world coordinate values covered by this axis (min,max of tiles)

    \subsubsection{CoordAxis1D.function}
      \textbf{vodml-id: CoordAxis1D.function} \newline
      \textbf{type: \hyperref[sect:CoordFunction1D]{caom2:CoordFunction1D}} \newline
      \textbf{multiplicity: 0..1} \newline
      pixel and world coordinate values covered by this axis (coordinates of every pixel)

  \subsection{CoordAxis2D}
  \label{sect:CoordAxis2D}
    a two-dimensional coordinate axis pair: quantity, values, errors; it is usually only necessary to specify one of the range, bounds, or function as they describe the world and pixel coordinate coverage at different levels of detail and the less detailed description is redundant (exception: when the range or bounds and function are both specified, the range/bounds is a subset of the pixels described by the function and denotes the valid pixels

    \subsubsection{CoordAxis2D.axis1}
      \textbf{vodml-id: CoordAxis2D.axis1} \newline
      \textbf{type: \hyperref[sect:Axis]{caom2:Axis}} \newline
      \textbf{multiplicity: 1} \newline
      first axis of the spatial coordinate system; usually longitude

    \subsubsection{CoordAxis2D.axis2}
      \textbf{vodml-id: CoordAxis2D.axis2} \newline
      \textbf{type: \hyperref[sect:Axis]{caom2:Axis}} \newline
      \textbf{multiplicity: 1} \newline
      second axis of the spatial coordinate system; usually latitude

    \subsubsection{CoordAxis2D.error1}
      \textbf{vodml-id: CoordAxis2D.error1} \newline
      \textbf{type: \hyperref[sect:CoordError]{caom2:CoordError}} \newline
      \textbf{multiplicity: 0..1} \newline
      position errors on the first axis

    \subsubsection{CoordAxis2D.error2}
      \textbf{vodml-id: CoordAxis2D.error2} \newline
      \textbf{type: \hyperref[sect:CoordError]{caom2:CoordError}} \newline
      \textbf{multiplicity: 0..1} \newline
      position errors on the second axis

    \subsubsection{CoordAxis2D.range}
      \textbf{vodml-id: CoordAxis2D.range} \newline
      \textbf{type: \hyperref[sect:CoordRange2D]{caom2:CoordRange2D}} \newline
      \textbf{multiplicity: 0..1} \newline
      pixel and world coordinate values covered by this axis (min,max)

    \subsubsection{CoordAxis2D.bounds}
      \textbf{vodml-id: CoordAxis2D.bounds} \newline
      \textbf{type: \hyperref[sect:CoordBounds2D]{caom2:CoordBounds2D}} \newline
      \textbf{multiplicity: 0..1} \newline
      pixel and world coordinate values covered by this axis (boundary)

    \subsubsection{CoordAxis2D.function}
      \textbf{vodml-id: CoordAxis2D.function} \newline
      \textbf{type: \hyperref[sect:CoordFunction2D]{caom2:CoordFunction2D}} \newline
      \textbf{multiplicity: 0..1} \newline
      pixel and world coordinate values covered by this axis (coordinates of every pixel

  \subsection{CoordBounds1D}
  \label{sect:CoordBounds1D}
    a one-dimensional sequence of reference coordinate ranges

    \subsubsection{CoordBounds1D.vertices}
      \textbf{vodml-id: CoordBounds1D.samples} \newline
      \textbf{type: \hyperref[sect:CoordRange1D]{caom2:CoordRange1D}} \newline
      \textbf{multiplicity: 1..*} \newline
      [TODO add description!]

  \subsection{CoordBounds2D}
  \label{sect:CoordBounds2D}
    a two-dimensional region in pixel and world coordinates; this can be used to specify a simple polygon boundary in pixel and world coordinates

    \subsubsection{CoordBounds2D.vertices}
      \textbf{vodml-id: CoordBounds2D.vertices} \newline
      \textbf{type: \hyperref[sect:Coord2D]{caom2:Coord2D}} \newline
      \textbf{multiplicity: 1..*} \newline
      the vertices of the polygon with implicit segment from the last vertex back to the first

  \subsection{CoordError}
  \label{sect:CoordError}
    coordinate error from FITS WCS

    \subsubsection{CoordError.crder}
      \textbf{vodml-id: CoordError.rnder} \newline
      \textbf{type: \hyperref[sect:ivoa]{ivoa:real}} \newline
      \textbf{multiplicity: 0..1} \newline
      [TODO add description!]

    \subsubsection{CoordError.csyer}
      \textbf{vodml-id: CoordError.syser} \newline
      \textbf{type: \hyperref[sect:ivoa]{ivoa:real}} \newline
      \textbf{multiplicity: 0..1} \newline
      [TODO add description!]

  \subsection{CoordFunction1D}
  \label{sect:CoordFunction1D}
    a one-dimensional (linear) WCS coordinate transformation function

    \subsubsection{CoordFunction1D.dimension}
      \textbf{vodml-id: CoordFunction1D.dimension} \newline
      \textbf{type: \hyperref[sect:ivoa]{ivoa:integer}} \newline
      \textbf{multiplicity: 1} \newline
      number of pixels along the axis

    \subsubsection{CoordFunction1D.refCoord}
      \textbf{vodml-id: CoordFunction1D.refCoord} \newline
      \textbf{type: \hyperref[sect:RefCoord]{caom2:RefCoord}} \newline
      \textbf{multiplicity: 1} \newline
      the reference pixel and world coordinate values

    \subsubsection{CoordFunction1D.delta}
      \textbf{vodml-id: CoordFunction1D.delta} \newline
      \textbf{type: \hyperref[sect:ivoa]{ivoa:real}} \newline
      \textbf{multiplicity: 1} \newline
      delta in world coordinate value (size of one pixel)

  \subsection{CoordFunction2D}
  \label{sect:CoordFunction2D}
    a two-dimensional (linear) WCS coordinate transformation function; this can be used to compute the world coordinates of every pixel

    \subsubsection{CoordFunction2D.dimension}
      \textbf{vodml-id: CoordFunction2D.dimension} \newline
      \textbf{type: \hyperref[sect:Dimension2D]{caom2:Dimension2D}} \newline
      \textbf{multiplicity: 1} \newline
      number of pixels along each axis of the two-dimensional space (FITS: NAXISi)

    \subsubsection{CoordFunction2D.refCoord}
      \textbf{vodml-id: CoordFunction2D.refCoord} \newline
      \textbf{type: \hyperref[sect:Coord2D]{caom2:Coord2D}} \newline
      \textbf{multiplicity: 1} \newline
      two-dimensional reference pixel and world coordinate values (FITS: CRPIXi, CRVALi)

    \subsubsection{CoordFunction2D.cd11}
      \textbf{vodml-id: CoordFunction2D.cd11} \newline
      \textbf{type: \hyperref[sect:ivoa]{ivoa:real}} \newline
      \textbf{multiplicity: 1} \newline
      two-dimensional scale and rotation (CD) matrix (FITS: CD1\_1)

    \subsubsection{CoordFunction2D.cd12}
      \textbf{vodml-id: CoordFunction2D.cd12} \newline
      \textbf{type: \hyperref[sect:ivoa]{ivoa:real}} \newline
      \textbf{multiplicity: 1} \newline
      two-dimensional scale and rotation (CD) matrix (FITS: CD1\_2)

    \subsubsection{CoordFunction2D.cd21}
      \textbf{vodml-id: CoordFunction2D.cd21} \newline
      \textbf{type: \hyperref[sect:ivoa]{ivoa:real}} \newline
      \textbf{multiplicity: 1} \newline
      two-dimensional scale and rotation (CD) matrix (FITS: CD2\_1)

    \subsubsection{CoordFunction2D.cd22}
      \textbf{vodml-id: CoordFunction2D.cd22} \newline
      \textbf{type: \hyperref[sect:ivoa]{ivoa:real}} \newline
      \textbf{multiplicity: 1} \newline
      two-dimensional scale and rotation (CD) matrix (FITS: CD2\_2)

  \subsection{CoordRange1D}
  \label{sect:CoordRange1D}
    a one-dimensional range of reference coordinates

    \subsubsection{CoordRange1D.start}
      \textbf{vodml-id: CoordRange1D.start} \newline
      \textbf{type: \hyperref[sect:RefCoord]{caom2:RefCoord}} \newline
      \textbf{multiplicity: 1} \newline
      [TODO add description!]

    \subsubsection{CoordRange1D.end}
      \textbf{vodml-id: CoordRange1D.end} \newline
      \textbf{type: \hyperref[sect:RefCoord]{caom2:RefCoord}} \newline
      \textbf{multiplicity: 1} \newline
      [TODO add description!]

  \subsection{CoordRange2D}
  \label{sect:CoordRange2D}
    a two-dimensional range of reference coordinates; this can be used to specify an axis-aligned bounding box in pixel and world coordinates

    \subsubsection{CoordRange2D.start}
      \textbf{vodml-id: CoordRange2D.start} \newline
      \textbf{type: \hyperref[sect:Coord2D]{caom2:Coord2D}} \newline
      \textbf{multiplicity: 1} \newline
      the two-dimensional reference coordinates with minimum longitude and latitude

    \subsubsection{CoordRange2D.end}
      \textbf{vodml-id: CoordRange2D.end} \newline
      \textbf{type: \hyperref[sect:Coord2D]{caom2:Coord2D}} \newline
      \textbf{multiplicity: 1} \newline
      the two-dimensional reference coordinates with maximum longitude and latitude

  \subsection{CustomWCS}
  \label{sect:CustomWCS}
    one-dimensional pixel and world coordinates describing a non-standard (custom) coordinate axis

    \subsubsection{CustomWCS.axis}
      \textbf{vodml-id: CustomWCS.axis} \newline
      \textbf{type: \hyperref[sect:CoordAxis1D]{caom2:CoordAxis1D}} \newline
      \textbf{multiplicity: 1} \newline
      description of the custom axis

  \subsection{Dimension2D}
  \label{sect:Dimension2D}
    dimension (number of pixels) for a two-dimensional axis

    \subsubsection{Dimension2D.naxis1}
      \textbf{vodml-id: Dimension2D.naxis1} \newline
      \textbf{type: \hyperref[sect:ivoa]{ivoa:integer}} \newline
      \textbf{multiplicity: 1} \newline
      [TODO add description!]

    \subsubsection{Dimension2D.naxis2}
      \textbf{vodml-id: Dimension2D.naxis2} \newline
      \textbf{type: \hyperref[sect:ivoa]{ivoa:integer}} \newline
      \textbf{multiplicity: 1} \newline
      [TODO add description!]

  \subsection{EnergyTransition}
  \label{sect:EnergyTransition}
    [TODO add description!]

    \subsubsection{EnergyTransition.species}
      \textbf{vodml-id: EnergyTransition.species} \newline
      \textbf{type: \hyperref[sect:ivoa]{ivoa:string}} \newline
      \textbf{multiplicity: 1} \newline
      TODO

    \subsubsection{EnergyTransition.transition}
      \textbf{vodml-id: EnergyTransition.transition} \newline
      \textbf{type: \hyperref[sect:ivoa]{ivoa:string}} \newline
      \textbf{multiplicity: 1} \newline
      TODO

  \subsection{ObservableAxis}
  \label{sect:ObservableAxis}
    an axis in the data (array) that varies by observable rather than coordinate; this axis is used when the data array containts values with different meaning in different subsets of the array (e.g. a row of pixels with wavelength values and a second row with flux values)

    \subsubsection{ObservableAxis.dependent}
      \textbf{vodml-id: ObservableAxis.dependent} \newline
      \textbf{type: \hyperref[sect:Slice]{caom2:Slice}} \newline
      \textbf{multiplicity: 1} \newline
      the part of the array containing the observable values

    \subsubsection{ObservableAxis.independent}
      \textbf{vodml-id: ObservableAxis.independent} \newline
      \textbf{type: \hyperref[sect:Slice]{caom2:Slice}} \newline
      \textbf{multiplicity: 0..1} \newline
      the part of the array containing coordinate values

  \subsection{PolarizationWCS}
  \label{sect:PolarizationWCS}
    one-dimensional pixel and world coordinates describing the polarization states

    \subsubsection{PolarizationWCS.axis}
      \textbf{vodml-id: PolarizationWCS.axis} \newline
      \textbf{type: \hyperref[sect:CoordAxis1D]{caom2:CoordAxis1D}} \newline
      \textbf{multiplicity: 1} \newline
      description of the polarization axis

  \subsection{RefCoord}
  \label{sect:RefCoord}
    a reference coordinate with a pixel and cooresponding world coordinate value

    \subsubsection{RefCoord.pix}
      \textbf{vodml-id: RefCoord.pix} \newline
      \textbf{type: \hyperref[sect:ivoa]{ivoa:real}} \newline
      \textbf{multiplicity: 1} \newline
      [TODO add description!]

    \subsubsection{RefCoord.val}
      \textbf{vodml-id: RefCoord.val} \newline
      \textbf{type: \hyperref[sect:ivoa]{ivoa:real}} \newline
      \textbf{multiplicity: 1} \newline
      [TODO add description!]

  \subsection{Slice}
  \label{sect:Slice}
    a one-dimensional subset of a two-dimensional array

    \subsubsection{Slice.axis}
      \textbf{vodml-id: Slice.axis} \newline
      \textbf{type: \hyperref[sect:Axis]{caom2:Axis}} \newline
      \textbf{multiplicity: 1} \newline
      description of the values within a the slice

    \subsubsection{Slice.bin}
      \textbf{vodml-id: Slice.bin} \newline
      \textbf{type: \hyperref[sect:ivoa]{ivoa:integer}} \newline
      \textbf{multiplicity: 1} \newline
      a constant-pixel value in the two-dimensional array that specifies the pixels in the slice

  \subsection{SpatialWCS}
  \label{sect:SpatialWCS}
    World Coordinate System (WCS) metadata for the position axes

    \subsubsection{SpatialWCS.axis}
      \textbf{vodml-id: SpatialWCS.axis} \newline
      \textbf{type: \hyperref[sect:CoordAxis2D]{caom2:CoordAxis2D}} \newline
      \textbf{multiplicity: 1} \newline
      description of the two-dimensional position axes

    \subsubsection{SpatialWCS.coordsys}
      \textbf{vodml-id: SpatialWCS.coordsys} \newline
      \textbf{type: \hyperref[sect:ivoa]{ivoa:string}} \newline
      \textbf{multiplicity: 0..1} \newline
      name of the coordinate system

    \subsubsection{SpatialWCS.equinox}
      \textbf{vodml-id: SpatialWCS.equinox} \newline
      \textbf{type: \hyperref[sect:ivoa]{ivoa:real}} \newline
      \textbf{multiplicity: 0..1} \newline
      equinox of the coordinate system

    \subsubsection{SpatialWCS.resolution}
      \textbf{vodml-id: SpatialWCS.resolution} \newline
      \textbf{type: \hyperref[sect:ivoa]{ivoa:real}} \newline
      \textbf{multiplicity: 0..1} \newline
      effective resolution of the data (FWHM of a point source); this is usually the value measured at the time of data acquisition

  \subsection{SpectralWCS}
  \label{sect:SpectralWCS}
    [TODO add description!]

    \subsubsection{SpectralWCS.axis}
      \textbf{vodml-id: SpectralWCS.axis} \newline
      \textbf{type: \hyperref[sect:CoordAxis1D]{caom2:CoordAxis1D}} \newline
      \textbf{multiplicity: 1} \newline
      description of the one-dimensional energy axis

    \subsubsection{SpectralWCS.specsys}
      \textbf{vodml-id: SpectralWCS.specsys} \newline
      \textbf{type: \hyperref[sect:ivoa]{ivoa:string}} \newline
      \textbf{multiplicity: 1} \newline
      reference frame for the spectral coordinate

    \subsubsection{SpectralWCS.ssysobs}
      \textbf{vodml-id: SpectralWCS.ssysobs} \newline
      \textbf{type: \hyperref[sect:ivoa]{ivoa:string}} \newline
      \textbf{multiplicity: 0..1} \newline
      reference frame that is constant over the range of the non-spectral world coordinates

    \subsubsection{SpectralWCS.ssyssrc}
      \textbf{vodml-id: SpectralWCS.ssyssrc} \newline
      \textbf{type: \hyperref[sect:ivoa]{ivoa:string}} \newline
      \textbf{multiplicity: 0..1} \newline
      reference frame for the velocity of the source (zsource)

    \subsubsection{SpectralWCS.restfrq}
      \textbf{vodml-id: SpectralWCS.restfrq} \newline
      \textbf{type: \hyperref[sect:ivoa]{ivoa:real}} \newline
      \textbf{multiplicity: 0..1} \newline
      rest frequency of the spectral feature of interest

    \subsubsection{SpectralWCS.restwav}
      \textbf{vodml-id: SpectralWCS.restwav} \newline
      \textbf{type: \hyperref[sect:ivoa]{ivoa:real}} \newline
      \textbf{multiplicity: 0..1} \newline
      rest wavelength of the spectral feature of interest

    \subsubsection{SpectralWCS.velosys}
      \textbf{vodml-id: SpectralWCS.velosys} \newline
      \textbf{type: \hyperref[sect:ivoa]{ivoa:real}} \newline
      \textbf{multiplicity: 0..1} \newline
      correction for the observatory's motion with respect to the barycenter

    \subsubsection{SpectralWCS.zsource}
      \textbf{vodml-id: SpectralWCS.zsource} \newline
      \textbf{type: \hyperref[sect:ivoa]{ivoa:real}} \newline
      \textbf{multiplicity: 0..1} \newline
      redshift of the source (relative to ssyssrc)

    \subsubsection{SpectralWCS.velang}
      \textbf{vodml-id: SpectralWCS.velang} \newline
      \textbf{type: \hyperref[sect:ivoa]{ivoa:real}} \newline
      \textbf{multiplicity: 0..1} \newline
      angle of true velocity from tangent to line of sight

    \subsubsection{SpectralWCS.bandpassName}
      \textbf{vodml-id: SpectralWCS.bandpassName} \newline
      \textbf{type: \hyperref[sect:ivoa]{ivoa:string}} \newline
      \textbf{multiplicity: 0..1} \newline
      telescope- or instrument-specific name for the energy band covered by the data; this is usually a filter name

    \subsubsection{SpectralWCS.transition}
      \textbf{vodml-id: SpectralWCS.transition} \newline
      \textbf{type: \hyperref[sect:EnergyTransition]{caom2:EnergyTransition}} \newline
      \textbf{multiplicity: 0..1} \newline
      description of the energy transition observed

    \subsubsection{SpectralWCS.resolvingPower}
      \textbf{vodml-id: SpectralWCS.resolvingPower} \newline
      \textbf{type: \hyperref[sect:ivoa]{ivoa:real}} \newline
      \textbf{multiplicity: 0..1} \newline
      ratio of wavelength to resolution (lambda/delta-lambda)

  \subsection{TemporalWCS}
  \label{sect:TemporalWCS}
    one-dimensional pixel and world coordinates describing the time axis

    \subsubsection{TemporalWCS.axis}
      \textbf{vodml-id: TemporalWCS.axis} \newline
      \textbf{type: \hyperref[sect:CoordAxis1D]{caom2:CoordAxis1D}} \newline
      \textbf{multiplicity: 1} \newline
      description of the time axis

    \subsubsection{TemporalWCS.timesys}
      \textbf{vodml-id: TemporalWCS.timesys} \newline
      \textbf{type: \hyperref[sect:ivoa]{ivoa:string}} \newline
      \textbf{multiplicity: 0..1} \newline
      time scale for the time coordinates

    \subsubsection{TemporalWCS.trefpos}
      \textbf{vodml-id: TemporalWCS.trefpos} \newline
      \textbf{type: \hyperref[sect:ivoa]{ivoa:string}} \newline
      \textbf{multiplicity: 0..1} \newline
      reference position for the time coordinates

    \subsubsection{TemporalWCS.mjdref}
      \textbf{vodml-id: TemporalWCS.mjdref} \newline
      \textbf{type: \hyperref[sect:ivoa]{ivoa:real}} \newline
      \textbf{multiplicity: 0..1} \newline
      base time offset; time coordinate values are relative to this

    \subsubsection{TemporalWCS.exposure}
      \textbf{vodml-id: TemporalWCS.exposure} \newline
      \textbf{type: \hyperref[sect:ivoa]{ivoa:real}} \newline
      \textbf{multiplicity: 0..1} \newline
      duration in time that the instrument was collecting data

    \subsubsection{TemporalWCS.resolution}
      \textbf{vodml-id: TemporalWCS.resolution} \newline
      \textbf{type: \hyperref[sect:ivoa]{ivoa:real}} \newline
      \textbf{multiplicity: 0..1} \newline
      smallest separation in time that can be distinguished